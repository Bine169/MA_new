\section{Introduction}

\subsection{Definition and aim of Geomarketing}
The term of Geomarketing has established more and more in the field of marketing within the last years.  A first approximation to the notion of Geomarketing was done in 1995 by \citeauthor{fruehling}. They had explained that Geomarketing is just a genus for several instruments within the field of marketing. This shows that Geomarketing is no methodology but rather a discipline. Although some definitions had occurred in the 90s, the first use of Geomarketing analysis went back to the 50s. Already 1952 the first map showing the purchasing power of Germany was published. In 1982 several companies were founded, which have offered tools and possibilities for their costumers to practise geographic analysis so that these researches got easier more and more. Consequently the comprehensive application of Geomarketing was born \cite{herter}. Within the early 90s Geomarketing described approaches and fundamentals of geographic analysis and the governance of marketing and distribution. The central idea of Geomarketing is that marketing composed of price, product, distribution and communication will be complemented by space. Consequently operating numbers can change dependent on the spatial location by spatial phenomena of productions and logistics.   


\begin{figureOwn}{Generic consideration of Geomarketing aspects complemented by space \cite{herter}}{images/geomarketing.jpg}\end{figureOwn}

Although the central idea of Geomarketing is known the definition of Geomarketing leads to some problems so that often incorrect explanations are written down. It is often readable that Geomarketing is a spatial analysis of the market using a geographic information system. But by particularly consideration it is recognizable that this statement is not correct because a geographic information system is just a tool which supports Geomarketing analysis to make them easier. \citeauthor{herter} tried to find a definition of Geomarketing which fits best compared to other definitions. They have defined that Geomarketing analyses current as well as potential markets considering spatial structures to make the planing of product sales more effective and to control them quantifiable. That means that all available information about the market are connected to a spatial reference system to make dependences, potentials and other properties visible. The application of Geomarketing analysis have their origins in the minimization of entrepreneurial risks by making the market more transparent and acting purposeful. In the course of this Geomarketing has established as a sub-discipline of the field of marketing. During the application of Geomarketing analysis several benefits can be achieved. By knowing potential costumers and competitors the marketing and distribution of products from a company can be done more dedicated so that efficiency enhancement and cost reduction can be caused. Additionally advantages to the competitors are achieved by the analyses. Furthermore inquests of the market may be helpful during the planing of new locations to determine a site with a high potential so that the risk of a malinvestment can be minimized. It is recognizable that as higher as the number of costumers is, none the worse the benefit of Geomarketing analyses are. During the surveys of the market several principles are utilized. One of them is the spatial factor of the market. Using spatial data of costumers, locations, branch offices etc. important dependences can be visualized. The spatial data are almost given using addresses. Additionally a spatial heterogeneity can be recognized during analyses that means that the market differentiates in space. in conclusion a third principle is generated by that fact. It describes the spatial segmentation of the market. It is recognizable that a subdivision into homogeneous market segments is possible. Consumers in the western part and the eastern part of Germany differentiate to each other, but an identification of costumers with similar affectations within a small range of space is possible. From this it follows the neighbourhood principle which explains that neighboured customers have a similar behaviour considering marketing aspects as product purchase. This fact has two reasons. On the one hand costumers with an analogical lifestyle life in the same space and consequently show common characters in costumer behaviour. On the other hand these people share the same infrastructure which takes influences to their purchasing habits as well. Additionally the distance to the location of a company affects the costumers in their decisions going to it or not. 
Geomarketing is based on three stacks: information of the market, geodata and analyses. 

\begin{figurevarSize}{Basic stacks of Geomarketing}{images/stacks.jpg}{0.5}\end{figurevarSize}

Market information are qualitative facts about costumers, competitors etc. within a regional (economic) zone. The data contain information about their socio-demographic, psychosocial, economic and consumption properties like income, product affinity, gender and household size \cite{tappert}. Geodata are information with a spatial reference like addresses, sales areas, locations and catchment areas. The boundaries of these regions may be administrative borders like federal states or townships as well as street sections, coordinates of houses and individual created areas per example a subdivision of postal code areas. By connecting market information with the help of geocoding to these geodata analyses are possible to obtain important knowledge about the market which are helpful to support companies in their marketing decisions. These facts show that Geomarketing is an instrument for analysing, planning, checking and controlling the market. In the meantime Geomarketing is grown up to one of the most important approaches within the field of Geomarketing to support companies while the accomplishment of their strategies. Consequently it is getting more and more essential to have systems providing functions and tools which making these analyses easier and more efficient. 

\subsection{microm Micromarketing-Systems and Consult GmbH}
Microm Micromarketing-Systems and Consult GmbH is a company in Germany which provide Geomarketing analyses to their costumers. It was founded in 1992 and since 1997 it is a subsidiary of the Creditreform. During the last decades microm grew up to one of the biggest providers of micromarketing and Geomarketing within Germany. It offers possibilities and tools to do anaylses of Geomarketing data. With the help of projects company owners can give their data to microm so that the analyses will be done by them. This approach offers the advantage that the company can use all the knowledge which is provides by the employees of microm to control further steps of their marketing decisions. Besides that procedure microm offers additionally a web tool to their costumers so that they can do the analyses by their self. The software is called mapChart Manager and is accessible with the help of a web browser like Firefox or Google Chrome. 

\begin{figureOwn}{Screenshot of the mapChart Manager}{images/mapchart.png}\end{figureOwn}

The advantage of a web tool like the mapChart Manager is that the users can have access to their data and maps from all over the world. Consequently sharing results and working independently from a computer and location makes the application of Geomarketing analyses more easily. The mapChart manager offers such functionality like the import of data, geocoding of addresses and do anaylses like catchment areas and driving distance zones. To do all that analyses a lot of data is indispensable  like routing networks or information about the behaviour of potential costumers. All these data are offered by the microm so that their costumers can buy the information they need. Doing so microm profits from their affiliation to the Creditreform, which collects costumer data from different resources among other things. As a subsidiary of the Creditreform the microm can sell all the information which the Creditreform have been collected.

\subsection{Motivation and Research Question}


\subsection{Methods}
blabla

\subsection{Outline}